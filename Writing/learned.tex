

\title{stuff}
\author{Joel Svensson}
\date{\today}

\documentclass[12pt]{article}
\setlength{\parindent}{0pt}
\setlength{\parskip}{1ex plus 0.5ex minus 0.2ex}
\begin{document}
\maketitle




\section{OpenGL}

\subsection{Positioning objects} 
Objects are positioned using the MODELVIEW matrix. Moving the 
``camera'' relative to the objects or the objects relative to 
the ``camera'' has the same effect. Therefore both these concepts
are managed by the MODELVIEW transformation matrix (hence its name). 

\begin{small}
\begin{verbatim}
glMatrixMode(GL_MODELVIEW);
\end{verbatim}
\end{small}       

The function {\tt gluLookAt()} positions the camera. If no position 
is specified (no call to {\tt gluLookAt()} the camera is placed 
at (0,0,0) looking down the negative $Z$ axis and up is towards positive $Y$
values.                                   

\begin{figure}
\begin{small}
\begin{verbatim}
glMatrixMode(GL_MODELVIEW);
glLoadIdentity();
glTranslatef(-1.0,0.0,-6.0);
glutSolidTeapot(0.5);
glLoadIdentity();
glTranslatef(1.0,0.0,-6.0);
glutSolidTeapot(0.5);
\end{verbatim}
\end{small} 
\label{Place two teapots next to eachother, one way}                   
\end{figure}

\begin{figure}
\begin{small}
\begin{verbatim}
glMatrixMode(GL_MODELVIEW);
glLoadIdentity();
glTranslatef(-1.0,0.0,-6.0);
glutSolidTeapot(0.5);
glTranslatef(2.0,0.0,0.0);
glutSolidTeapot(0.5);
\end{verbatim}
\end{small} 
\label{Place two teapots next to eachother, another way. Second teapot 
       is placed relative to the first one}                   
\end{figure}


\subsection{Projection transformation} 
Think of this as the properties of the lens of the camera. 
This transformation is specified in the PROJECTION matrix. 

\begin{small}
\begin{verbatim}
glMatrixMode(GL_PROJECTION);
\end{verbatim}
\end{small}       

                       
\subsection{Shaders}
 (Much of this is differnt in OpenGL 3.x and GLSL 1.5) 

Code here uses functions available from OpenGL version 2.0 and up: 

\begin{small}
\begin{verbatim}
 void setShaders() {
	
  char *vs,*fs;
	
  vertexShader   = glCreateShader(GL_VERTEX_SHADER);
  fragmentShader = glCreateShader(GL_FRAGMENT_SHADER);	
	
  vs = textFileRead("Shaders/basic_v.glsl");
  fs = textFileRead("Shaders/basic_f.glsl");
	
  const char * vv = vs;
  const char * ff = fs;
	
  //             handle          #strings   code    lengths 
  glShaderSource(vertexShader,   1,         &vv,    NULL);
  glShaderSource(fragmentShader, 1,         &ff,    NULL);
	
  free(vs);free(fs);
	
  glCompileShader(vertexShader);
  glCompileShader(fragmentShader);
	
  p = glCreateShader();
		
  glAttachShader(p,vertexShader);
  glAttachShader(p,fragmentShader);
	
  glLinkProgram(p);
  glUseProgram(p);
  
  printGLError("setShader:");
}


\end{verbatim}
\end{small}               

{\tt glShaderSource} takes the code as an ({\tt char **}) (why? when does that help?).   


Silly vertex program  
\begin{small}
\begin{verbatim} 
void main() {		
  gl_Position = gl_ProjectionMatrix * gl_ModelViewMatrix * gl_Vertex;
}
\end{verbatim}
\end{small}               
 
Silly fragment program.     
\begin{small}
\begin{verbatim} 
uniform vec4 Color; 

void main() {
  gl_FragColor = Color; /* vec4(1.0,1.0,1.0,1.0);*/
}

\end{verbatim}
\end{small}    
The {\tt uniform vec4 Color} specifies an input to the shader called Color.

\subsection{GLEW} 

The OpenGL Extensions Wrangler. Use this to be able to access  
openGL features for versions of opengl higher than 1.1. 
(I understand there may be Linux vs Windows differences here)

Include {\tt GL/glew.h} and link with {\tt lGLEW}.

for example this enables the use of functions such as: 

\begin{small} 
\begin{verbatim}
glCreateShader();
glShaderSource();
glCompileShader();
glCreateProgram();
glAttachShader();
glLinkProgram();
glUseProgram();
\end{verbatim}
\end{small}

\subsection{GLUT}

Important functions: 
\begin{itemize}
\item{void glutInit(int *argc,char **argv)} 
\item{void glutInitDisplayMode(unsigned int mode)}
\item{void glutInitWindowSize(int width, int height)}
\item{int glutCreateWindow(char *name})
\item{glutMainLoop()}                                                        
\end{itemize}

Basic GLUT example:
\begin{small} 
\begin{verbatim}
// C99 
// gcc -std=c99 main.c -o main -lglut
#include <stdio.h>
#include <stdlib.h>

#include <GL/glut.h> 

void display(void) {
  glClearColor(0.0,0.0,0.0,0.0);
  glClear(GL_COLOR_BUFFER_BIT);

  glutWireTeapot(0.5);

  glutSwapBuffers();
}

int main(int argc, char **argv) {
  
  int win;
  
  //Initialize 
  glutInit(&argc,argv);

  //Set graphics 
  glutInitDisplayMode(GLUT_RGB | GLUT_DOUBLE);

  //Create window 
  glutInitWindowSize(640,480); 
  win = glutCreateWindow("Gfx");

  //Register callbacks
  glutDisplayFunc(display);

  glutMainLoop();
}
\end{verbatim}
\end{small}

\section{SSE Intrinsics}

How does SSE fit in with Graphics programming ? 
Questions I want answered: 
\begin{itemize} 
\item{SOA or AOS} Graphics examples found online seem to imply that 
          AOS is to be used. For SIMD (SSE) SOA seems to be preferred. 
\end{itemize}


Important header files:
\begin{itemize}
\item{mmintrin.h}  MMX intrinsics
\item{xmmintrin.h} SSE intrinsics
\item{emmintrin.h} SSE2 intrinsics
\item{pmmintrin.h} SSE3 intrinsics
\item{tmmintrin.h} SSSE3 intrinsics (what is this ?) 
\item{ammintrin.h} SSE4A intrinsics
\item{smmintrin.h} SSE4.1 intrinsics
\item{nmmintrin.h} SSE4.2 intrinsics
\item{bmmintrin.h} SSE5A intrinsics

\end{itemize} 

Detect SSE version: 

\begin{small}
\begin{verbatim}
#define cpuid(func,ax,bx,cx,dx)\
        __asm__ __volatile__ ("cpuid":\
        "=a" (ax), "=b" (bx), "=c" (cx), "=d" (dx) : "a" (func));
\end{verbatim}
\end{small}

\begin{small}
\begin{verbatim}
// on return v and m contains largest supported sse version
void sse_version(int *v, int *m) {
  
  int shifts[5] = {25,26,1,19,20}; 
  
  int a,b,c,d; 
  
  cpuid(1,a,b,c,d); 

  for (int i = 1; i < 6; i ++) {
    if (i < 2) { 
      if (d & (1 << shifts[i-1]))  *v = i; 
    }
    else
      if (c & (1 << shifts[i-1])) {
       *v = i > 4 ? 4 : i; 
       *m = *m + i >= 4 ? 1 : 0;
      }
  }
}
\end{verbatim}
\end{small}

\begin{small}
\begin{verbatim}

int has_sse() {
  int a,b,c,d; 

  a = b = c = d = 0; 
  
  cpuid(1,a,b,c,d); 
  
  return (d & (1 << 25));
  
}

int has_sse2() {
  int a,b,c,d; 

  a = b = c = d = 0; 
  
  cpuid(1,a,b,c,d); 
  
  return (d & (1 << 26));
  
}

int has_sse3() {
  int a,b,c,d; 

  a = b = c = d = 0; 
  
  cpuid(1,a,b,c,d); 
  
  return (c & 1);
  
}

int has_sse4_1() {
  int a,b,c,d; 

  a = b = c = d = 0; 
  
  cpuid(1,a,b,c,d); 
  
  return (c & (1 << 19));
  
}

int has_sse4_2() {
  int a,b,c,d; 

  a = b = c = d = 0; 
  
  cpuid(1,a,b,c,d); 
  
  return (c & (1 << 20));
  
}


\end{verbatim}
\end{small}


\appendix
\section{Data}

\begin{small}
\begin{verbatim}
// Cube all triangles specified. (easy to stick in a VBO)
// as a Triangle strip would be more efficient 
Vector3f cube_v2[36] =
  {
    {-1.0,-1.0,1.0 }, {-1.0,-1.0,-1.0 },  {-1.0,1.0,-1.0 },  //#0
    {-1.0,1.0,-1.0 }, {-1.0,1.0,1.0 },    {-1.0,-1.0,1.0 },  //#1
    {-1.0,1.0,1.0 },  {-1.0,1.0,-1.0 },    {1.0,1.0,-1.0 },  //#2
    {1.0,1.0,-1.0 },  {1.0,1.0,1.0 } ,    {-1.0,1.0,1.0},    //#3
    {1.0,1.0,1.0 },   {1.0,1.0,-1.0 },    {1.0,-1.0,-1.0 },  //#4
    {1.0,-1.0,-1.0 }, {1.0,-1.0,1.0 },    {1.0,1.0,1.0 },    //#5
    {1.0,-1.0,1.0 },  {1.0,-1.0,-1.0 },   {-1.0,-1.0,-1.0 }, //#6
    {-1.0,-1.0,-1.0 }, {-1.0,-1.0,1.0 },  {1.0,-1.0,1.0 },   //#7 
    {1.0,-1.0,-1.0 },  {1.0,1.0,-1.0 },   {-1.0,1.0,-1.0 },  //#8
    {-1.0,1.0,-1.0 },  {-1.0,-1.0,-1.0 }, {1.0,-1.0,-1.0 },  //#9
    {1.0,1.0,1.0 },    {1.0,-1.0,1.0 },   {-1.0,-1.0,1.0 },  //#10
    {-1.0,-1.0,1.0 },  {-1.0,1.0,1.0 },   {1.0,1.0,1.0 }     //#11
  }; 


// normals per vertex (duplicated face normals) 
Vector3f cube_n2[36] = 
  { 
    {-1.0, 0.0, 0.0}, {-1.0, 0.0, 0.0}, {-1.0, 0.0, 0.0},  
    {-1.0, 0.0, 0.0}, {-1.0, 0.0, 0.0}, {-1.0, 0.0, 0.0},
    {0.0, 1.0, 0.0}, {0.0, 1.0, 0.0}, {0.0, 1.0, 0.0}, 
    {0.0, 1.0, 0.0}, {0.0, 1.0, 0.0}, {0.0, 1.0, 0.0}, 
    {1.0, 0.0, 0.0}, {1.0, 0.0, 0.0}, {1.0, 0.0, 0.0},
    {1.0, 0.0, 0.0}, {1.0, 0.0, 0.0}, {1.0, 0.0, 0.0},  
    {0.0, -1.0, 0.0}, {0.0, -1.0, 0.0}, {0.0, -1.0, 0.0}, 
    {0.0, -1.0, 0.0}, {0.0, -1.0, 0.0}, {0.0, -1.0, 0.0}, 
    {0.0, 0.0, -1.0}, {0.0, 0.0, -1.0}, {0.0, 0.0, -1.0}, 
    {0.0, 0.0, -1.0}, {0.0, 0.0, -1.0}, {0.0, 0.0, -1.0},
    {0.0, 0.0, 1.0}, {0.0, 0.0, 1.0}, {0.0, 0.0, 1.0},
    {0.0, 0.0, 1.0}, {0.0, 0.0, 1.0}, {0.0, 0.0, 1.0} 
  };
\end{verbatim}
\end{small}

\end{document}
This is never printed
